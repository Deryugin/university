\documentclass[20pt]{article}
\usepackage{anyfontsize}
\usepackage[utf8]{inputenc}
\usepackage[russian]{babel}
\usepackage{amssymb}
\usepackage{mathtools}
\usepackage[left=2cm,right=2cm,
	    top=2cm,bottom=2cm,bindingoffset=0cm]{geometry}

\title{Задание 2 (Вариант 1)}
\author{
	Дерюгин Денис, студент 661-й учебной группы
}
%\date{\today}

\begin{document}
\large{
\maketitle
Будет применяться точный критерий Фишера.

$X$ --- тип применяемого лекарства (1 --- препарат А, 0 --- плацебо), $Y$ --- наличие симптомов (1 --- нет симптомов, 0 --- есть).

\begin{center}
    \begin{tabular}{ | c | c | c | c |}
        \hline
        $X \backslash Y$ & 1 & 0 & сумма \\ \hline
        1 & c = 6 & d = 3 & 9 \\ 
        0 & a = 3 & b = 6 & 9 \\ \hline
        сумма & 9 & 9 & n = 18\\ 
        \hline
    \end{tabular}
\end{center}

$p_0 = P\{Y = 0 | X= 0 \}, p_1 = P\{Y=0 | X=1\}$.  \\

$H_0$: вероятность наличия симптомов не зависит от типа принимаемого лекарства, $p_0=p1$.

$H_1$: $p_0 \neq p_1$. \\

Вычислим уровень значимости $P_{a+c}^a$ --- вероятность иметь симптомы, принимая препарат A ($X=1, Y=1$). \\


$P_{a+c}^a = \dfrac{(a+b)!(c+d)!(a+c)!(b+d)!}{a!b!c!d!n!} \approx 0.1451$. \\

$0.1451 > p = 0.05$, следовательно, принимается гипотеза $H_0$, т.е. статистически значимых различий нет.

}
\end{document}
