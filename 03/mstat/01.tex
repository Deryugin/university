\documentclass[20pt]{article}
\usepackage{anyfontsize}
\usepackage[utf8]{inputenc}
\usepackage[russian]{babel}
\usepackage{amssymb}
\usepackage{mathtools}
\usepackage[left=2cm,right=2cm,
	    top=2cm,bottom=2cm,bindingoffset=0cm]{geometry}

\title{Задание 1 (Вариант 1)}
\author{
	Дерюгин Денис, студент 661-й учебной группы
}
%\date{\today}

\begin{document}
\large{
\maketitle
$X_1$ --- измерение до рекламы, $X_2$ --- после неё.


\begin{center}
    \begin{tabular}{ | c | c | c | c |}
        \hline
        $X_1 \backslash X_2$ & да & нет & сумма \\ \hline
        да & 5 & 0 & 5 \\ 
        нет & 6 & 4 & 10\\ \hline
        сумма & 11 & 4 & 15\\ 
        \hline
    \end{tabular}
\end{center}

$b$ и $c$ --- элементы таблицы сопряжённости, не лежащие на главной диагонали (т.е. те случаи, когда результат измерения изменился).

$H_0$: маргинальные распределения всех исходов совпадают, $p_b = p_c$.
$H_1$: $p_b \neq p_c$.



Точная статистика критерия Мак-Немара: \\ \\
$\alpha_* = 2 \sum_{i=0}^{min(0,6)}C_{0 + 6}^i \dfrac{1}{2^{0 + 6}} = 2 C^0_6 \dfrac{1}{64} = 2^{-5} = 0,03125$.\\

Таким образом, $\alpha_* < \alpha = 0.05$. \\ \\
% Статистический критерий Мак-Немара:\\$\chi^2 = \dfrac{(b-c)^2}{b+c} = 6$.\\ \\

Коррекция Эдвардса:\\ $\chi^2_* = \dfrac{(|b-c|-1)^2}{b+c} = 4\dfrac{1}{6}$.\\

$\chi^2_{0.05,1} \approx 6,5706 > 4\dfrac{1}{6}$

Доверительный уровень вероятности $p=P\{\chi^2 > \chi^2_* = 4\dfrac{1}{6}\} \approx 0.04123$.

Гипотеза $H_0$ отвергается.
}
%http://r-analytics.blogspot.ru/2012/11/blog-post.html
\end{document}
