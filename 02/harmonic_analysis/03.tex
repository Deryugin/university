\documentclass{article}
\usepackage{anyfontsize}
\usepackage[utf8]{inputenc}
\usepackage[russian]{babel}
\usepackage{amssymb}
\usepackage{mathtools}
\usepackage[left=2cm,right=2cm,
	    top=2cm,bottom=2cm,bindingoffset=0cm]{geometry}
\newcommand\proofend{\begin{flushright}$\blacksquare$\end{flushright}}
\title{Задание 3}
\author{
	Дерюгин Денис, студент 561-й учебной группы
}
\date{\today}


\begin{document}
\maketitle
\large{
	\textbf{Условие:} пусть $X$ --- спектр Фурье сигнала $x$ (т.е. $X = \mathcal{F}_N(x)$).\\ Сопоставим ему сигнал $y$:\\
	
	$y(k) = \left\{\begin{matrix}
			X(0), & if\;k = 0 \\ 
			X(N - k), & otherwise 
		\end{matrix}\right.$\\
	\\

	Доказать, что $x = \dfrac{1}{N} \mathcal{F}_N(y)$.\\
	\\

	\textbf{Доказательство:} По формуле обращения:\\
	$x(j) = \dfrac{1}{N}\sum\limits_{k = 0}^{N-1} X(k)\omega_N^{kj}$.\\
	\\ \\
	Помня, что $\omega_N^i$ --- N-периодическая функция, выполним преобразования:\\
	$\mathcal{F}_N(y)(k) = \sum\limits_{j = 0}^{N - 1} y(j) \omega_N^{-kj} = y(0) \omega_N^0 + \sum\limits_{j = 1}^{N - 1} y(j) \omega_N^{-kj} = X(0) \omega_N^0 + \sum\limits_{j = 1}^{N - 1} X(N - j) \omega_N^{-kj} =\\ \\ \\= X(0) \omega_N^0 + \sum\limits_{p = 1}^{N - 1} X(p) \omega_N^{-k(N - p)} = \sum\limits_{p = 0}^{N - 1} X(p) \omega_N^{kp}$.\\ \\ \\
	Следовательно,\\
	$x(j) = \dfrac{1}{N}\sum\limits_{k = 0}^{N - 1} X(k) \omega_N^{jk} = \dfrac{1}{N} \mathcal{F}_N(y)(j)$.
	
	\proofend

}
\end{document}
