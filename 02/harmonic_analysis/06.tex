\documentclass{article}
\usepackage[utf8]{inputenc}
\usepackage[russian]{babel}
\usepackage{amssymb}
\usepackage{mathtools}
\usepackage[left=2cm,right=2cm,
	    top=2cm,bottom=2cm,bindingoffset=0cm]{geometry}
\newcommand\proofend{\begin{flushright}$\blacksquare$\end{flushright}}
\title{Задание 6}
\author{
	Дерюгин Денис, студент 561-й учебной группы
}
\date{\today}


\begin{document}
\maketitle
\large{
	\textbf{Условие:} Доказать: $x \in \mathbb{C}$ --- чётный $\Leftrightarrow X$ --- вещественный.\\
	\\
	\textbf{Доказательство:} \\
	\textbf{Необходиомость:} $x$ --- чётный $\Rightarrow x(-k) = \overline{x}(k) \Rightarrow x(N - k) = \overline{x}(k)$.\\
	$2X(j) = \sum\limits_{k = 0}^{N - 1} x(k) \omega_N^{-kj} + \sum\limits_{l = 1}^{N - 1} x(N - l) \omega_N^{-(N - l)j} = \sum\limits_{k = 0}^{N - 1} x(k) \omega_N^{-kj} + \sum\limits_{l = 1}^{N - 1} \overline{x}(l) \omega_N^{lj} = $\\
	$= \sum\limits_{k = 1}^{N - 1} \bigg [ x(k) \omega_N^{-kj} + \overline{x(k) \omega_N^{-kj}} \bigg ] = \sum\limits_{k = 0}^{N - 1} 2Re(x(k) \omega_N^{-kj}) \in \mathbb{R}$.

	\textbf{Достаточность:} $x(-k) = \dfrac{1}{N} \sum\limits_{j = 0}^{N - 1} X(j) \omega_N^{-kj} = \dfrac{1}{N} \sum\limits_{j = 0}^{N - 1} \overline{X(j) \omega_N^{kj}} = \overline{\dfrac{1}{N} \sum\limits_{j = 0}^{N - 1} X(j) \omega_N^{kj}} = \overline{x(k)}$.

	\proofend
}
\end{document}
