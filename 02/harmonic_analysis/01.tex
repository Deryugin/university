\documentclass[38pt]{article}
\usepackage{anyfontsize}
\usepackage[utf8]{inputenc}
\usepackage[russian]{babel}
\usepackage{amssymb}
\usepackage{mathtools}
\usepackage[left=2cm,right=2cm,
	    top=2cm,bottom=2cm,bindingoffset=0cm]{geometry}

\title{Задание 1}
\author{
	Дерюгин Денис, студент 561-й учебной группы
}
\date{\today}


\begin{document}
\maketitle
\large{
	\textbf{Условие:} Доказать, что $\forall x \in \mathbb{C}_{N} \; \exists y, z : x = y + z$, где $y$ --- чётный сигнал, а $z$ --- нечётный (то есть, $\forall j \in \mathbb{Z} : y(-j) = \overline{y(j)}, z(-j) = -\overline{z(j)}$).

\textbf{Доказательство:} Допустим, что такие сигналы существуют. Тогда $\forall j \in \mathbb{Z}$:\\

$\left\{
\begin{matrix}
	x(j) & = y(j) + z(j); \\ 
	\\
	x(-j) & = \overline{y(j)} - \overline{z(j)};
\end{matrix}
\right.$\\
\\

$\left\{
\begin{matrix}
	x(j) & = y(j) + z(j); \\ 
	\\
	\overline{x(-j)} & = y(j) - z(j);
\end{matrix}
\right.$\\
\\

$\left\{
\begin{matrix}
	\dfrac{x(j) + \overline{x(-j)}}{2} = y(j); \\
	\\
	\dfrac{x(j) - \overline{x(-j)}}{2} = z(j);
\end{matrix}
\right.$\\
\\

Проверим чётность $y$:\\
$y(-j) = \dfrac{x(-j) + \overline{x(j)}}{2} = \dfrac{\overline{\overline{x(-j)} + {x(j)}}}{2} = \overline{y(j)}.$\\
\\

Проверим нечётность $z$:\\
$z(-j) = \dfrac{x(-j) - \overline{x(j)}}{2} = -\dfrac{\overline{{x(j)} - \overline{x(-j)}}}{2}= -\overline{z(j)}.$\\
\\

Проверим, что $x = y + z$:\\
$y(j) + z(j) = \dfrac{x(j) + \overline{x(-j)}}{2} + \dfrac{x(j) - \overline{x(-j)}}{2} = x(j).$

\raggedleft{}{$\blacksquare$}

}
\end{document}
