\documentclass{article}
\usepackage[utf8]{inputenc}
\usepackage[russian]{babel}
\usepackage{amssymb}
\usepackage{mathtools}
\usepackage[left=2cm,right=2cm,
	    top=2cm,bottom=2cm,bindingoffset=0cm]{geometry}
\newcommand\proofend{\begin{flushright}$\blacksquare$\end{flushright}}
\title{}
\author{
	Дерюгин Денис, студент 561-й учебной группы
}
\date{\today}


\begin{document}
\maketitle
\large{
	\section*{Задача 9}

	$X(k) = \sum\limits_{j=0}^{N-1}sin\dfrac{\pi j}{N}\omega_N^{-kj} = \sum\limits_{j=0}^{N-1}sin\dfrac{\pi j}{N}(cos\dfrac{2\pi k j}{N} - isin\dfrac{2\pi k j}{N}) = $\\
	$= \sum\limits_{j=0}^{N-1} sin\dfrac{(2k+1)j\pi}{N} - cos\dfrac{\pi j}{N}sin\dfrac{2\pi k j}{N} + i\bigg (cos \dfrac{(2k+1)j\pi}{N} - cos\dfrac{\pi j}{N}cos\dfrac{2\pi k j}{N} \bigg ) =$ \\
	$= \sum\limits_{j=0}^{N-1} i\omega_{2N}^{(2k+1)\pi j} - cos\dfrac{\pi j}{N} \bigg (sin \dfrac{2kj\pi}{N} +i sin \dfrac{2kj\pi}{N} \bigg) = \sum\limits_{j=0}^{N-1} i\omega_{2N}^{(2k+1)\pi j} - i\omega_{2N}^{(2k-1) \pi j} - sin\dfrac{\pi j}{N} \omega_{2N}^{-kj} \Rightarrow$ \\ \\ \\
	$X(k) = \dfrac{1}{2i} \sum\limits_{j=0}^{N-1} \omega_{2N}^{(2k+1)\pi j} - \omega_{2N}^{(2k-1) \pi j} = ...$ (геометрическая прогрессия) \\
	$... = \dfrac{1}{2i} \bigg ( \dfrac{1 - \omega_{2N}^{-(2k - 1)N}}{1 - \omega_{2N}^{-(2k-1)}} - \dfrac{1 - \omega_{2N}^{-(2k + 1)N}}{1 - \omega_{2N}^{-(2k+1)}} \bigg ) = \dfrac{1}{i} \bigg ( \dfrac{1}{1 - \omega_{2N}^{-(2k-1)}} - \dfrac{1}{1 - \omega_{2N}^{-(2k+1)}} \bigg ) =$\\
	$= \dfrac{1}{i} \bigg ( \ctg \dfrac{(2k+1)\pi}{2N} - ctg\dfrac{(2k-1)\pi}{2N} \bigg )$
	\proofend

	\section*{Задача 10}
	$m\sum\limits_{q=0}^{n-1}\delta_N(k - qm) = \left\{\begin{matrix}
		0, & k\;mod\;m \neq 0 \\
		m\sum\limits_{q=0}^{n-1}\delta_n(a-q), & k = ma, a \in \mathbb{Z}
	\end{matrix}\right. = \left\{\begin{matrix}
		0, & k\;mod\;m \neq 0 \\
		m, & k = ma, a \in \mathbb{Z}
	\end{matrix}\right. = m\delta_m(k)$.

	$X(k) = \sum\limits_{j = 0}^{N - 1} \omega_N^{-kj} \sum\limits_{p=0}^{m-1}\delta_N(j - pn) = \sum\limits_{b=0}^{m-1} \omega_N^{-kbn} \sum\limits_{p=0}^{m-1}\delta_m(b - p) = \sum\limits_{b=0}^{m-1} \omega_m^{-kb} = m\delta_m(k)$.
	\proofend
	\pagebreak

	\section*{Задача 12}
	По теореам о циклической корреляции и о свёртке:\\
	$\mathcal{F}_n(R_{uu}) = \mathcal{F}_n ( x \ast y ) \overline{\mathcal{F}_n(x \ast y)} = XY\overline{XY} = X\overline{X}Y\overline{Y} = \mathcal{F}_n(R_{xx})\mathcal{F}_n(R_{yy}) = \mathcal{F}_n(R_{xx} \ast R_{yy}) \Rightarrow$ \\
	$\Rightarrow R_{uu} = R_{xx} \ast R_{yy}$
	\proofend

	\section*{Задача 13}
	По равенству Парсеваля и теореме о циклической свёртке:\\
	$\| R_{xy} \|^2 = \langle R_{xy}, R_{x,y}\rangle  = \dfrac{1}{N} \langle \mathcal{F}_n(R_{xy}), \mathcal{F}_n(R_{xy})\rangle  = \dfrac{1}{N} \langle X\overline{Y}, X\overline{Y}\rangle  = ...$ \\(по определнию скалярного произведения)\\$... = \dfrac{1}{N} \langle X\overline{X}, Y\overline{Y}\rangle  = \langle R_{xx}, R_{yy}\rangle $
	\proofend

	\section*{Задача 14}
	Пусть $u = x \ast y$.\\
	Из задачи 12: $R_{uu}(j) = (R_{xx} \ast R_{yy})(j) = \sum\limits_{k = 0}^{N - 1} R_{xx}(k) R_{yy}(j - k).$\\
	$x, y$ --- $\delta$-коррелированные $\Rightarrow R_{xx}(j) = R_{yy}(j) = 0 \;\;\forall j \in [1 ... N - 1]$.\\ \\ \\
	$R_{uu}(j) = R_{xx}(0) R_{yy}(j) = \left\{\begin{matrix}
		0, & j \in [1..N-1] \\
		R_{xx}(0)R_{yy}(0), & otherwise

	\end{matrix}\right. \Rightarrow R_{uu} $ --- $\delta$-коррелированный.
	\proofend

	\section*{Задание 16}
	Представим число $k = k_0 + 2k_1 + 2^2k_2 + ... + 2^sk_s$ в виде вектора $(k_0, k_1, ..., k_s)$.\\
	Так как $k$ и $j < \dfrac{N}{2}$, $k_s = j_s = 0$.\\
	$2k = 2k_0 + 2^2k_1 + ... + 2^sk_{s-1} = (0, k_0, ..., k_{s-1})$.\\
	$2j = 2j_0 + 2^2j_1 + ... + 2^sj_{s-1} = (0, j_0, ..., j_{s-1})$.\\ \\
	$\hat v_{2k}(j) = v_{rev(2k)}(j) = \{rev(2k), j\}_s = \{(k_{s-1}, ..., k_0, 0), (j_0, j_1, ..., j_{s-1}, 0)\} = \sum\limits_{p = 0}^{s-1}j_pk_{s-1-p}$.\\
	$\hat v_{k}(2j) = v_{rev(k)}(2j) = \{rev(k), 2j\}_s = \{(0,k_{s-1}, ..., k_0), (0, j_0, j_1, ..., j_{s-1})\} = \sum\limits_{p = 0}^{s-1}j_pk_{s-1-p}$.\\
	\proofend
	\pagebreak

	\section*{Задача 15}
	Используем теорему: $A_k[k, j] = v_k(j)$.\\
	\textbf{База индукции:} $k = 1, N = 2^k = 2$.
	$A_1 = \bigl(\begin{smallmatrix}
		1 & 1\\
		1 & -1
	\end{smallmatrix}\bigr)
	$.\\
	$v_0(2 - 1) = 1$.\\
	$v_0(2 - 1 - 0) = 1 = v_0(0)$.\\
	$v_0(2 - 1 - 1) = 1 = v_0(1)$.\\ \\
	$v_1(2 - 1) = -1$.\\
	$v_1(2 - 1 - 0) = -1 = v_1(0)$.\\
	$v_1(2 - 1 - 1) = 1 = -v_1(1)$.\\ \\
	\textbf{Индукционный переход:} Пусть для всех $k' < k$ утверждение верно, $N = 2^k$.\\
	Рассмотрим $v_l$ такие, что $l < \dfrac{N}{2}$.\\
	$v_l(a) = v_l(a + \dfrac{N}{2})$.\\ \\ \\
	Если $j < \dfrac{N}{2}$, то $v_l(N - 1 - j) = v_l(\dfrac{N}{2} - 1 - j) = \left\{\begin{matrix}
		v_l(j), & v_l(\dfrac{N}{2} - 1 - j) = 1 \\
		-v_l(j), & v_l(\dfrac{N}{2} - 1 - j) = -1
	\end{matrix}\right.$ \\ \\ по индукционному предположению.\\ \\ \\
	Если $j >= \dfrac{N}{2}$, то $v_l(N - 1 - j) = v_l(N - 1 - j + \dfrac{N}{2}) = \left\{\begin{matrix}
		v_l(j), & v_l(N - 1 - j) = 1 \\
	-v_l(j), & v_l(N - 1 - j) = -1 \end{matrix} \right.$. \\ \\ по индукционному предположнию.\\ \\ \\
	Пусть теперь $l >= \dfrac{N}{2}$, тогда $v_l(N - 1) = -v_l(\dfrac{N}{2} - 1)$.\\
	Пусть $v_l(N - 1) = 1$. Тогда $v_l(\dfrac{N}{2} - 1) = -1$.\\
	Если $j < \dfrac{N}{2}$, то $v_l(j) = -v_l(\dfrac{N}{2} - 1 - j) = v_l(\dfrac{N}{2} - 1 - j + \dfrac{N}{2}) = v_l(N - 1 - j)$.\\
	Если $j \geq \dfrac{N}{2}$, то $v_l(j) = -v_l(j - \dfrac{N}{2}) = v_l(\dfrac{N}{2} - 1 - (j - \dfrac{N}{2})) = v_l(N - 1 - j)$. \\ \\
	Пусть $v_l(N - 1) = -1$. Тогда $v_l(\dfrac{N}{2} - 1) = 1$.\\
	Если $j < \dfrac{N}{2}$, то $v_l(j) = v_l(\dfrac{N}{2} - 1 - j) = -v_l(\dfrac{N}{2} - 1 - j + \dfrac{N}{2}) = -v_l(N - 1 - j)$.\\
	Если $j \geq \dfrac{N}{2}$, то $v_l(j) = -v_l(j - \dfrac{N}{2}) = -v_l(\dfrac{N}{2} - 1 - (j - \dfrac{N}{2})) = -v_l(N - 1 - j)$. \\ \\
	\proofend
	\pagebreak


\end{document}
