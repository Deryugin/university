\documentclass{article}
\usepackage[utf8]{inputenc}
\usepackage[russian]{babel}
\usepackage{amssymb}
\usepackage{mathtools}
\usepackage[left=2cm,right=2cm,
	    top=2cm,bottom=2cm,bindingoffset=0cm]{geometry}
\newcommand\proofend{\begin{flushright}$\blacksquare$\end{flushright}}
\title{}
\author{
	Дерюгин Денис, студент 561-й учебной группы
}
\date{\today}

\begin{document}
\maketitle
\setcounter{MaxMatrixCols}{20}
\large {
\section*{Задача 27}
Используем определение корреляции и то, что $x$ --- $N$-периодическая функция.\\
$R_{xx}(j) = \sum\limits_{k=0}^{N-1} x(k) \overline{x}(k - j).$\\
Если $j = 0$, требуемое равенство следует из того, что каждое слагаемое будет иметь вид $x(k)\overline{x}(k)$, и оно равно самому себе сопряжённому. Следовательно, это верно и для всей суммы, т.е. $R_{xx}(0) = \overline{R_{xx}}(0)$.\\

Пусть теперь $j \neq 0$.\\
$\sum\limits_{k = 0}^{N - 1} x(k) \overline{x}(k - j) =$
$\sum\limits_{l = -j}^{N - 1 - j} x(l + j) \overline{x}(l) =$
$\sum\limits_{l = -j}^{- 1} x(l + j) \overline{x}(l) + \sum\limits_{l = 0}^{N - 1 - j} x(l + j) \overline{x}(l) = $ \\
$= \sum\limits_{l = N - j}^{N - 1} x(l + j) \overline{x}(l) + \sum\limits_{l = 0}^{N - j - 1} x(l + j) \overline{x}(l) = $
$\sum\limits_{l = 0}^{N - 1} \overline{x}(l) x(l + j) = $
$\overline{\sum\limits_{l = 0}^{N - 1} x(l) \overline{x}(l + j)} = \overline{R_{xx}}(-j)$
\proofend

\section*{Задача 28}
Пусть $x(j) = \omega_N^{aj}, y(j) = \omega_N^{bj}$. При этом $a, b \in [0 : N - 1]$.\\
Учтём, что $\delta_N(t) = \dfrac{1}{N}\sum\limits_{q=0}^{N-1}\omega_N^{-qt}$.\\
$R_{xy}(j) = $
$\sum\limits_{k = 0}^{N - 1} x(k) \overline{y} (k - j) =$
$\sum\limits_{k = 0}^{N - 1} \omega_N^{ak} \overline{\omega_N^{b(k-j)}} =$
$\sum\limits_{k = 0}^{N - 1} \omega_N^{ak} \omega_N^{-b(k - j)} =$
$\sum\limits_{k = 0}^{N - 1} \omega_N^{k(a - b) + bj} = $\\
$= \omega_N^{bj}\sum\limits_{k = 0}^{N - 1} \omega_N^{k(a - b)} = $
$\omega_N^{bj}\dfrac{1}{N}\delta_N(a - b).$\\

Учитывая то, что $a, b \in [0 : N -1]$, $(a - b) \in [-N + 1 : N - 1]$, значит, $R_{xy}(j) = 0 \forall j$, если $a \neq b$.

\proofend
\pagebreak
\section*{Задача 29}

Пусть $x(j) = \delta_N{j - a}, y(j) = \delta_N^{j - b}$. При этом $a, b \in [1 : N - 1]$.\\

$R_{xy}(j) = $
$\sum\limits_{k = 0}^{N - 1} x(k) \overline{y} (k - j) =$
$\sum\limits_{k = 0}^{N - 1} \delta_N(k - a) \overline{\delta_N} (b - k + j).$\\

Для всех $k \neq a\;\;\delta_N(k - a)$ будет равно $0$, так как $k - a \in [-N+1:N-1]$. Значит, останется только одно слагаемое.\\

$R_{xy}(j) = \delta_N(a - a) \overline{\delta_N} (b - a + j) = $
$\delta_N (b - a + j).$\\
Следовательно, $R_{xy}$ обращается в $1$ при $j = a - b$. То есть, функции $x$ и $y$ не являются некоррелированными.

\proofend

\section*{30}
Будем использовать теорему о корреляции:\\
$\mathcal{F}_N(R_{xy}) = X\overline{Y}$. \\

Обозначим $a(j) = R_{xw}(j)$, $b(j) = R_{yz}(j)$.\\

$\mathcal{F}_N(R_{ab}) =$
$\mathcal{F}_N(R_{xw}) \overline{\mathcal{F}_N}(R_{yz}) =$
$X\overline{W} \overline{Y\overline{Z}} = $
$X\overline{Y} Z \overline{W} = $
$\mathcal{F}_N(R_{xy}) Z\overline{W}.$

По условию, $R_{xy}$ --- тождественный $0$. Следовательно, по определению, $\mathcal{F}_N(R_{xy}) = 0$.\\

Значит, и $\mathcal{F}_N(R_{xy}) Z \overline{W}$ --- также тождественный $0$.\\

Значит, $\mathcal{F}_N(R_{ab})(j) = 0 \;\;\forall j \in [0 : N - 1]$. По формуле обращения, $R_{ab}(j) = 0 \forall j \in [0 : N - 1]$.
\proofend
}
\end{document}
